\section{Limitations of Polynomial Models for Temperature Dependence \label{sec::T_lims}}

The regression model (\ref{eqn::regression}) interpolates the exponential temperature dependence of the term
$k_{scr}\Gamma$ using a quadratic. This approximation introduces an error but results in a linear in parameters model
with unique solution under least-squares for interpolation. The model identified from the data has a strong dependence
on the temperature characteristics of the experimental data. Fitting a quadratic temperature model for the response of a
nonlinear function of temperature in least-squares sense involves inverting the following matrix, $\Lambda$, which is
only a function of the temperature sampling and multiplying it to a vector that is a function of the output of nonlinear function at the temperature samples.
\begin{align}
        \Lambda = \frac{1}{N}\bm{\sum T_i^4 & \sum T_i^3 & \sum T_i^2\\
                                 \sum T_i^3 & \sum T_i^2 & \sum T_i^1\\
                                 \sum T_i^2 & \sum T_i^1 & N\\  }  \qquad i \in & \lrf{1, 2, \hdots, N}
\end{align}
Thus, one way to ensure that the parameter estimates are comparable across data sets is to require that the sufficient statistics, which are solely functions of temperature, remain the same (or approximately so).
\begin{align}
        %\min{\lr{T_i}},  \max{\lr{T_i}} &       \label{eqn::T_suff_1}\\
        \bm{\frac{\sum T_i}{N}, & \frac{\sum{T_i^2}}{N}, & \frac{\sum{T_i^3}}{N}, & \frac{\sum{T_i^4}}{N}} \qquad i \in & \lrf{1, 2, \hdots, N}     \label{eqn::T_suff_2}
\end{align}
Although these conditions are stringent, they can be satisfied through a judicious selection of data segments. Given
that the operating temperature range of the catalyst is finite, the minimum and maximum temperatures can be defined
accordingly. Subsequently, data segments may be included or excluded based on whether the corresponding sufficient
statistics satisfy the prescribed criteria. This can be posed as an integer programming problem.
