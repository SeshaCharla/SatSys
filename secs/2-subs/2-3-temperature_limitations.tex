\subsection{Limitations of Polynomial Models for Temperature Dependence}

The regression model (\ref{eqn::regression}) interpolates the exponential temperature dependence of the term
$k_{scr}\Gamma$ using a quadratic. This approximation introduces an error but results in a linear in parameters model
with unique solution under least-squares for interpolation. The model identified from the data has a strong dependence
on the temperature characteristics of the experimental data. Thus, for parameter estimates to be comparable across data
sets the following sufficient statistics that are sole functions of temperature must be close (lemma-\ref{lemma::T_suff}):
\begin{align}
        \min{\lr{T_i}},  \max{\lr{T_i}} &       \label{eqn::T_suff_1}\\
        \frac{\sum T_i}{N}, \frac{\sum{T_i^2}}{N}, \frac{\sum{T_i^3}}{N}, \text{ and } \frac{\sum{T_i^4}}{N} \qquad i \in & \lrf{1, 2, \hdots, N}       \label{eqn::T_suff_2}
\end{align}
Although these conditions are stringent, they can be satisfied through a judicious selection of data segments. Given
that the operating temperature range of the catalyst is finite, the minimum and maximum temperatures can be defined
accordingly. Subsequently, data segments may be included or excluded based on whether the corresponding sufficient
statistics satisfy the prescribed criteria. Furthermore, if the prediction error becomes significantly large, the model can be partitioned to enable switching between distinct temperature zones.
