%% 1. The NO_x reduction dynamics are different when the Catalyst is saturted as it is independent of the urea-dosing.
%====
Considering the reaction chamber as a control volume (Figure-\ref{fig:plug-flow}), the dynamic model is developed based
on the molar conservation of species at the inlet and outlet separated by a residence-time interval. By assuming a
zero-order hold on the inlet species over the sampling time, the equations relating inlet and outlet concentrations at
the residence-time scale are extended to the sampling-time scale. The residence time can be parameterized as a function of mass flow rate $\lr{F}$ assuming that the effect of the density variation on residence time is insignificant.
\begin{align}
    \tau(k) &= \frac{\rho V_{scr}}{F(k)} = \frac{\tau_0}{F(k)}
    \label{eqn::res_time}
\end{align}
Let, $\lrf{\bullet}$ be the number of moles and $\lrb{\bullet}$ be the concentration of a given species $\bullet$. We have the molar conservation of $NO_x$ across the control volume $(V_{scr})$ within one residence time $(\tau)$:
%====
\begin{multline}
        \mol{NO_x}^{out} (t + (i+1) \tau) =
                \mol{NO_x}^{in} (t + i \tau) \\
                + V_{scr} \int_0^{\tau} \frac{d}{dt} \con{NO_x}^{scr}dt
        \label{eqn::nox_bal}
\end{multline}
%===
The number of moles entering at the inlet and leaving at the outlet can be written as a function of the concentration, residence time and mass flow rate $(F)$:
%===
\begin{align}
    \mol{NO_x}^{in/out} = \con{NO_x}^{in/out} \tau \lr{\frac{F}{\rho}} = \con{NO_x}^{in/out} V_{scr}
\end{align}
%===
Using Eiley-Rideal mechanism for Standard SCR reaction $\lrf{1}$:
\begin{align}
    \frac{d}{dt} \con{NO_x}^{scr}dt = -k_{scr} \con{NH_3}^{ads} \con{NO_x}^{in}
\end{align}
%===
Rewriting (\ref{eqn::nox_bal}) in terms of concentrations and integrating to residence time:
\begin{multline}
        \con{NO_x}^{out} (t + (i+1) \tau) =
                \con{NO_x}^{in} (t + i \tau) \\
                -\tau k_{scr} \con{NH_3}^{ads} \con{NO_x}^{in}
        \label{eqn::nox_bal_con}
\end{multline}
%====
Summing the equations from (\ref{eqn::nox_bal_con}) for  $i = 0$ to $n-1$, where $n$ is the total number of residence times within one sampling period $(n= t_s/\tau)$ and using the zero-order hold assumptions at the inlet (\ref{eqn::zero_in}) and outlet (\ref{eqn::zero_out}) concentrations, we get equation (\ref{eqn::nox_avg}).
\begin{align}
    \con{NO_x}^{in} (t + i \tau) &= \con{NO_x}^{in}(t) \quad \forall i \in [0, n-1]  \label{eqn::zero_in}\\
    \con{NO_x}^{out} (t + (i+1) \tau) &= \con{NO_x}^{out}(t + t_s) \quad \forall i \in [0, n-1] \label{eqn::zero_out}
\end{align}
Let, tailpipe $NO_x$ concentration be $x_1(k+1) = \con{NO_x}^{out}(t+t_s)$ and inlet $NO_x$ concentration be $u_1(k) = \con{NO_x}^{in}(t)$, we get,
\begin{multline}
  \underbrace{\con{NO_x}^{out}(t+t_s)}_{x_1(k+1)} = \underbrace{\con{NO_x}^{in}(t)}_{u_1(k)} \\
                - \tau(t) k_{scr}(t) \con{NO_x}^{in}(t) \underbrace{\lrf{\frac{1}{n} \sum_{i = 0}^{n-1} \con{NH_3}^{ads}(t + i \tau)}}_{\gamma(k)}
        \label{eqn::nox_avg}
\end{multline}
%
% ======================================================================================================================
%
The quantity $\gamma(k)$ is the average concentration of ammonia adsorbed on the catalyst's surface across the $n$ residence times between $k$ and $k+1$ samples. $\gamma(k)$ is bounded by the total concentration of voids on the surface of the catalyst, $\Gamma$, which decrease with increase in temperature.
\begin{align}
    0 \leq \gamma(k) \leq \Gamma \qquad \forall k
\end{align}
%===
The unbounded dynamics of $\gamma$, $\gamma^{ub}(k)$ is a nonlinear function dependent on concentration of adsorbed ammonia, urea dosing $\lr{u_{inj}}$, and the inlet $NO_x$ concentration from the previous time step.
\begin{align}
    \gamma^{ub}(k) &= g_{\gamma} \lr{ \gamma(k-1), u_{inj}(k-1), \con{NO_x}^{in}(k-1) }
\end{align}
The actual expression, which is beyond the scope of this work, can be derived using the similar approach used for
deriving $NO_x$ process dynamics (\ref{eqn::nox_avg}). $\gamma(k)$ can be described using the following piecewise
function:
\begin{align}
    \gamma(k) &=
    \begin{cases}
        \gamma^{ub}(k) & \text{if } \quad 0 \leq \gamma^{ub}(k) \leq \Gamma \\
        \Gamma         & \text{if } \quad \gamma^{ub}(k) > \Gamma \qquad \text{[Saturation]}\\
        0              & \text{if } \quad \gamma^{ub}(k) < 0 \qquad \text{[Empty]}
    \end{cases}
\end{align}
%===
Let, $\eta(k)$ be the $NO_x$ reduction during a given time step $k$,
\begin{align}
    \eta (k+1) &= u_1(k) - x_1(k+1)
\end{align}
Introducing above definition and the residence time expression (\ref{eqn::res_time}) into the $NO_x$ process dynamics model (\ref{eqn::nox_avg}), we get,
\begin{align}
    \eta (k+1) &=   \frac{u_1(k)}{F(k)} \tau_0 \; k_{scr}(T(k)) \; \gamma (T(k)) \label{eqn::flow_scaled_nox_eff}
\end{align}
%===
For $NO_x$ process dynamics under catalyst saturation, $\gamma(k) = \Gamma$.
\begin{align}
    \implies \eta_{sat} (k+1) &=  \frac{u_1(k)}{F(k)} \tau_0 \; k_{scr}(T(k)) \; \Gamma (T(k))
        \label{eqn::nox_mdl}
\end{align}
%===
The above equation~(\ref{eqn::nox_mdl}) shows that the tailpipe $NO_x$ concentration becomes independent of urea dosing under catalyst saturation. Polynomial approximations of Arrhenius temperature dependence of the rate constant, $k_{scr}$ and the concentration of viable voids, $\Gamma$ (\cite{nova2014urea},\cite{ciardelli2004scr},\cite{joo2008study}), are used to get a linear in parameters model for the $NO_x$ process dynamics.
\begin{align}
    k_{scr}(T) &= A_{scr} e^{\lr{-\frac{E_{scr}}{RT}}} \approx \sum k_i T^i  \qquad i = 0, 1, \hdots
    \label{eqn::rate_const} \\
    \Gamma(T) &= S_1 e^{-S_2 T} \approx \sum r_i T^i \qquad i = 0, 1, \hdots
    \label{eqn::gamma}
\end{align}
%===
The above approximate models for temperature introduce dependence of the parameter estimates based on temperature ranges
in the samples. For numerical stability \cite{press2003numerical}, the given temperature range is mapped to $[-1,1]$ and
Chebyshev polynomial bases \cite{trefethen2019approximation} are used instead of standard polynomials. Incorporating
equations (\ref{eqn::res_time}) and (\ref{eqn::gamma}) into the $NO_x$ process dynamic model (\ref{eqn::nox_mdl}),
\begin{align}
    \eta_{sat} (k+1) &= \phi_{sat}^T(T(k)) \; \theta_{sat}
    \label{eqn::regression} \\
    %===
    \phi_{sat} (k) &= \frac{u_1(k)}{F(k)} \bm{2\lr{\frac{T-T_0}{T_r}}^2-1 & \frac{T-T_0}{T_r} & 1}
    \label{eqn::phi_def} \\
    %===
    \text{where, } \quad T_0 &= \frac{T_{max} + T_{min}}{2}, \quad T_r = \frac{T_{max} - T_{min}}{2} \notag
\end{align}
