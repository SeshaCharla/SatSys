\section{SCR-ASC SYSTEM DYNAMICS UNDER CATALYST SATURATION}
The SCR-ASC subsystem in the diesel engine after-treatment systems involves reacting flow on the top of a catalyst which
adsorbs ammonia from urea decomposition. The reaction kinetics start with the decomposition of urea into ammonia
followed by its adsorption onto the surface of the catalyst. The $NO_x$ entering the chamber is reduced by this adsorbed
ammonia. The Eley-Rideal reaction mechanism \cite{yuan2015diesel}, \cite{hsieh2011development}, \cite{nova2014urea} is
considered for modeling the SCR reaction kinetics, where one reactant $(NO_x)$ is gaseous, and the other is adsorbed on
the catalyst surface $(NH_3)$. \cite{nova2014urea} lists all the reactions that take place inside the
SCR-ASC chamber. Ammonia oxidation (AMOX) happens both on the surfaces of the SCR catalyst and the ASC catalyst in the
mechanism that is similar, but SCR catalyst favors $NO_x$ reduction while ASC is specifically designed for ammonia
oxidation.
In order to keep the model order reasonably low and parameters identifiable, only the following three prominent reactions are considered:
\begin{enumerate}
    \item Standard SCR reaction:
    \begin{align*}
        4 NH_3 ^{ads} + 4 NO + O_2 &\xrightarrow[]{k_{scr}} 4 N_2 + 6 H_2O %\label{eqn::std_scr}
        & \lrf{1}
    \end{align*}
    \item Ammonia Oxidation:
    \begin{align*}
        4 NH_3^{ads} + 3 O_2 &\xrightarrow[]{k_{oxi}} 2 N_2 + 6 H_2O %\label{eqn::amox}
        & \lrf{2}
    \end{align*}
    \item Ammonia Adsorption/Desorption:
        \begin{align*}
            NH_3 + \Theta_{free} &\xrightleftharpoons[k_{des}]{k_{ads}} NH_3^{ads}
            %\label{eqn::ads}
            & \lrf{3}
        \end{align*}
\end{enumerate}

$\indent$ The ammonia oxidation (AMOX) reactions are similar on SCR and ASC catalysts. First, the gaseous ammonia
adsorbs onto the catalyst system. Then, the adsorbed ammonia oxidizes into $N_2$ or $NO_x$ based on the temperature and
other conditions. As the AMOX reactions on both catalysts are similar, it is not possible to distinguish the origin of
the products from the outlet measurements alone unless measurements between SCR and ASC sections are available.

Thus, oxidation of adsorbed ammonia on both SCR and ASC catalysts can be combined into a single reaction, lumping
the rate constants and concentrations of the products into parameters and states of a single oxidation reaction. Further
the nitrogen selectivity of that single AMOX reaction is assumed to be $100\%$. This is valid for temperatures greater
than $225 \lx{^o}{C}$ \cite{jain2023diagnostics}, i.e., the temperature range of interest for most of the test and road
conditions.

This aggregation of reactions results in errors in parameter estimates, specifically parameters containing rate constant for the $NO_x$ reduction $\lr{k_{scr}}$ will have higher (bounded) uncertainty as it becomes dependent on the nitrogen selectivity of AMOX reaction.
