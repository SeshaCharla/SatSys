\section{Model Paramters Capture Catalyst Aging}
The paramters of the saturated system a function of the rate-constant for SCR reaction and the maximum surface
concentration of vaible voids on the catalyst. The adsorption cite concentration is hypothesized to change with aging of
the catalyst allowing the parameters to capture the aging of the catalyst. This is demonstrated using the test-cell
experiments on a new (de-greened) and hydro-thermally aged catalyst with the same capacity. The predicted $NO_x$
reduction for an aged catalyst is found to be statistically lower than the predicted $NO_x$ reduction in a de-greened
catalyst (Figure-\ref{fig::aging_diff}).
\begin{figure}[H]
        \centering
        \includegraphics[width=\figWidth]{./figs/5_aging/AgingDiff.png}
        \caption{Predicted $\eta_{sat}$ for Aged and Degreened Catalyst}
        \label{fig::aging_diff}
\end{figure}
%===
When the $\eta_{sat}$ response is nromalized with respect to the inlet $NO_x$ concentration and flow-rate, we end up
with a quantity that is only a function of themperature and represents the product of rate-constant and maximum surface
coverage, i.e., from \ref{eqn::regression}:
\begin{align}
        \alpha_{sat} &= \frac{F(k)}{u_1(k)} \eta_{sat}(k) = \tau_0 k_{scr}\Gamma = \phi_T^T(k)\theta_{sat}\\
        \text{where,} \quad \phi_T^T(k) &= \bm{2\lr{\frac{T-T_0}{T_r}}^2-1 & \frac{T-T_0}{T_r} & 1}
\end{align}
%===
This quantity, $\alpha_{sat}$ is estimatible and changes with aging of the catalyst based on the hypothesis of changing
adsorption cite concentration. The change in $\alpha_{sat}$ with aging is demonstrated using the test-cell experiments.
\begin{figure}[H]
        \centering
        \includegraphics[width=\figWidth]{./figs/5_aging/aging_factor_ssd.png}
        \caption{$\hat \alpha_{sat}$ from RMC FTIR Data}
        \label{fig::alpha_sat_ssd}
\end{figure}
\begin{figure}[H]
        \centering
        \includegraphics[width=\figWidth]{./figs/5_aging/aging_factor_iod.png}
        \caption{$\hat \alpha_{sat}$ from RMC $NO_x$ Sensor Data}
        \label{fig::alpha_sat_iod}
\end{figure}
%===
Thus, $\theta_{sat}$ capture the aging of the catalyst. Hence, the catalyst aging detection problem can be posed as a
hypothesis testing problem with null hypothesis $\lr{\mathcal{H}_0}$ corresponding to a degreened catalyst and
alternative hypothesis $\lr{\mathcal{H}_1}$ corresponding to an aged catalyst. Since, the distribution of the error
(half-normal) is assumed to be known for both hypotheses and differ by the unknown (but estimatible) parameter
$\theta_{sat}$, the following Wald test \cite{wald1943tests} is proposed to check if the parameter estimates for the
given data set correspond to an aged catalyst or a known degreened catalyst. Decide $\mathcal{H}_1$ if,
\begin{align}
        T_w  &= \lr{\hat \theta_{sat} - \theta_{sat}^{dg}}^T I\lr{\hat \theta_{sat}} \lr{\hat \theta_{sat} - \theta_{sat}^{dg}} > \beta
\end{align}
$\theta_{sat}^{dg}$ is the parameter estimates for the degreened catalyst under same input conditions i.e.,
temperature, flow-rate and urea-dosing rate ranges. The threshold $\beta$ is decided based on the experimental results.
For RMC and FTP data sets, the value of the test-statistic is tabulated in \ref{tab::Tw_tst}.
% Table of test-statistic
\begin{table}[H]
        \centering
        \caption{Wald Test Results On the Test Cell Data}
        \label{tab::Tw_tst}
        \begin{tabular}{l c c}
        \hline \hline
        Test & $T_w$ from FTIR & $T_w$ from $NO_x$ sensor \\\hline \hline
        DG RMC 1      & 0.262 & 5.539\\
        DG RMC 2      & 0.000 & 0.000\\
        DG RMC 2      & 6.709 & 5.457\\
        Aged RMC      & 47.156 & 17.685\\
        \hline
        DG Hot FTP 1  & 0.49 & 0.81\\
        DG Hot FTP 2  & 3.26 & 1.98\\
        DG Hot FTP 3  & 0.00 & 0.00\\
        Aged Hot FTP  & 4.84 & 11.44\\
        \hline \hline
        \end{tabular}
\end{table}
The distribution (Non-central $\chi^2$ asymptotically) of the test-statistic under aging is different for RMC and FTP
test conditions due to the differences in the ranges of inputs (temperature, flow-rate and urea-dosing rate) experienced
by the catalyst during the tests. Based on the results, a reasonabley discerning threshold for Hot-FTP test conditions
is found to be $4.00$ and for RMC test conditions is found to be $10.00$. Further investigation with more data is needed
to arrive at the probability of detection and false alarm for these thresholds. The proposed Wald test detects the aged
catalyst for both FTIR and $NO_x$ sensor measurements under both test conditions.
