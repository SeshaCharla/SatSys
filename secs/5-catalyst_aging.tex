\section{Model Paramters Capture Catalyst Aging}
The paramters of the saturated system a function of the rate-constant for SCR reaction and the maximum surface
concentration of vaible voids on the catalyst. The adsorption cite concentration is hypothesized to change with aging of
the catalyst allowing the parameters to capture the aging of the catalyst. This is demonstrated using the test-cell
experiments on a new (de-greened) and hydro-thermally aged catalyst with the same capacity. The predicted $NO_x$
reduction for an aged catalyst is found to be statistically lower than the predicted $NO_x$ reduction in a de-greened
catalyst.

% Put the relavent plots here

Consequently, the following Wald test \cite{wald1943tests} is proposed to check if the parameter estimates for the given data set correspond to an aged catalyst ($\mathcal{H}_1$) or a known degreened catalyst ($\mathcal{H}_0$). The test-statistic is given as:
\begin{align}
        T_w \lr{\hat \theta_{sat}} &= \lr{\hat \theta_{sat} - \theta_{sat}^{dg}}^T I\lr{\hat \theta_{sat}} \lr{\hat \theta_{sat} - \theta_{sat}^{dg}} > \beta
\end{align}
$\theta_{sat}^{dg}$ is the parameter estimates for the degreened catalyst under same input conditions i.e.,  temperature, flow-rate and urea-dosing rate ranges. The threshold $\beta$ is decided based on the experimental results. For RMC and FTP data sets, the value of the test-statistic is tabulated in \ref{tab::Tw_tst}.

% Table of test-statistic

\begin{table}[H]
        \centering
        \caption{Wald Test On the Test Cell Data}
        \label{tab::Tw_tst}
        \begin{tabular}{l c c}
        \hline \hline
        Test & $T_w$ from FTIR & $T_w$ from $NO_x$ sensor \\\hline \hline
        DG RMC 1      & 0.262 & 5.539\\
        DG RMC 2      & 0.000 & 0.000\\
        DG RMC 2      & 6.709 & 5.457\\
        Aged RMC      & 47.156 & 17.685\\
        \hline
        DG Hot FTP 1  & 0.49 & 0.81\\
        DG Hot FTP 2  & 3.26 & 1.98\\
        DG Hot FTP 3  & 0.00 & 0.00\\
        Aged Hot FTP  & 4.84 & 11.44\\
        \hline \hline
        \end{tabular}
\end{table}

Based on the results, a reasonabley discerning threshold for FTP test condistions is found to be $\_$.
