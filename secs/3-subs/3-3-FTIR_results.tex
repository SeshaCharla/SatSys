The parameter estimates using FTIR measurements from the test-cell experiments are tabulated in
Table-\ref{tab::parm_FTIR}. The parameters show consistency across repeated RMC (Ramped Mode Cycle) and hot-FTP (Federal
Test Procedure) tests on the same catalyst condition validating the estimation approach. However, the parameters vary
significantly between RMC and hot-FTP tests. This is due to the differences in the temperature ranges experienced by the
catalyst during the tests, which affects the Arrhenius temperature dependence of the rate constant and the concentration
of viable voids on the catalyst surface. The temperature range of the data considered during the RMC tests is between
240°C and 360°C, while for the Hot-FTP tests it is between  200°C and 300°C.
\begin{table}[ht]
        \centering
        \caption{Parameter Estimates using FTIR Data}
        \label{tab::parm_FTIR}
        \begin{tabular}{l c c c c}
                \hline \hline
                Test & $\hat \theta_1 $ & $\hat \theta_2 $ & $\hat \theta_3 $ & $\sigma$\\ \hline \hline
                        DG RMC 1  & -0.63 & 0.55 & 38.74 & 1.14\\
                        DG RMC 2  & -0.82 & 0.69 & 38.61 & 1.08\\
                        DG RMC 2  & -0.80 & 1.12 & 38.78 & 1.10\\
                        Aged RMC  & -1.31 & 1.89 & 36.94 & 1.01\\ \hline
                        DG Hot FTP 1 & -2.14 & -5.80 & 39.53 & 1.35\\
                        DG Hot FTP 2 & -3.45 & -7.99 & 38.05 & 1.30\\
                        DG Hot FTP 3 & -3.37 & -6.42 & 38.88 & 1.33\\
                        Aged Hot FTP & -4.54 & -8.01 & 37.00 & 1.31\\
                \hline\hline
        \end{tabular}
\end{table}
