\section{Maximum Likelihood Estimate Derivation \label{sec::mle_derivation}}
Consider,
\begin{align}
        \varepsilon_\eta(k) &= \eta_{sat}(k) - \eta\lr{k}  \quad
        \text{or,}\quad \varepsilon_\eta(k) = \eta_{sat}(k) - \eta_y{k} \label{eqn::varepsilon_eta}\\
        \implies& \varepsilon_\eta(k) = \phi^T \theta_{sat} - \eta\lr{k} \label{eqn::eta_regression}
\end{align}
based on the measurement sensors. Assuming $\varepsilon_\eta$ follows an exponential distribution with the rate parameter, $\lambda$,
\begin{align}
        \varepsilon_\eta &\backsim \text{Exp}(\lambda);\qquad
        p(\varepsilon_\eta;\lambda) = \begin{cases} \lambda e^{-\lambda \varepsilon_\eta} & \varepsilon_\eta \geq  0\\
                                                        0 & \varepsilon_{\eta} < 0
                                        \end{cases}
                                        \label{eqn::exp_pdf}
\end{align}
we have the likelihood and log-likelihood functions given the data and parameter estimate $\theta_{sat}$ and $\lambda$:
\begin{align}
        p(\pmb \varepsilon_{\eta};\theta_{sat}, \lambda) &= \prod_k  \lambda e^{-\lambda \varepsilon_\eta (k)} & \varepsilon_\eta(k) \geq  0 \\
        \ln \lr{p(\pmb \varepsilon_{\eta};\theta_{sat}, \lambda)} &= N \ln \lambda -\lambda \sum_k \varepsilon_\eta (k) & \varepsilon_\eta(k) \geq  0
\end{align}
where, N is the total sample size. For, maximum likelihood estimate of $\lambda$,
\begin{align}
        \frac{\partial\ln \lr{p(\pmb \varepsilon_{\eta};\theta_{sat}, \lambda)}}{\partial \lambda} = 0 \qquad
        \implies \lambda = \frac{N}{\sum_k \varepsilon_\eta} \quad \varepsilon_\eta \geq 0
\end{align}
So, the log-likelihood functions becomes:
\begin{align}
        \ln \lr{p(\pmb \varepsilon_{\eta};\theta_{sat})}&= \lrb{N \ln N - N} - \ln \sum_k \varepsilon_{\eta}(k) & \varepsilon_\eta(k) \geq  0
\end{align}
Thus we have the maximum likelihood estimate $\theta_{sat}$ as,
\begin{align}
       \hat \theta_{sat} &= \argmax_{\theta_{sat}} \ln \lr{p(\pmb \varepsilon_{\eta};\theta_{sat}, \lambda)}
       = \argmin_{\theta_{sat}} \sum_{k}\varepsilon_\eta (k), \varepsilon_\eta(k) \geq  0 \notag \\
       %===
       &=\argmin_{\theta_{sat}}  \lrf{\sum_{k} \phi_{sat}(k) \theta_{\sat}}, \: s.t \: \phi_{sat}(k) \theta_{\sat} \geq \eta_{\sat} (k) \: \forall k
\end{align}
which is the linear programming problem (\ref{eqn::lin_prog}).
