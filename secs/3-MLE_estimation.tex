\section{Maximum Likelihood Estimation of the Model Parameters}

The catalyst switches between saturation mode and normal mode based on states that are not completely controllable experimentally. The bounding condition on the saturated system's response (\ref{eqn::eta_bound}) can be used for detecting segments of the data where the catalyst's operation is normal or saturated. As the saturated catalyst's response forms a tight upper bound to the actual system's response, the parameters of the systems will be the solution to the linear programming problem that minimizes the area under the curve with the lower bound as the response of the actual system.
\begin{align}
\text{minimize } \, \sum \eta_{sat}(k) \quad
\text{subject to } \, \lr{\eta_{sat}(k) \geq \eta(k)} \quad \forall  k \label{eqn::lin_opt_1}
\end{align}
%===
The above linear programming problem is feasible as $\eta_{sat}, \eta \geq 0$ always. The solution provides the system's
parameter estimates under ideal conditions, i.e., in the absence of model structure uncertainties and measurement noise.
When the $NO_x$ sensor data is used, $\eta(k)$ is replaced with $\eta_y(k)$ as the bounding condition
(\ref{eqn::eta_y_bound}) still holds. Writing equation (\ref{eqn::lin_opt_1}) using regression vector (\ref{eqn::regression}) and introducing the constraints on the parameters (\ref{eqn::theta_cond}), we have the linear programming problem for estimating the parameters of the catalyst:
\begin{align}
        \text{minimize} &\qquad \pmb 1^T \lrb{\pmb \Phi_{sat}^T \pmb \theta_{sat}}      \label{eqn::lin_prog}\\
        \text{subject to} &\qquad \pmb \Phi_{sat} \pmb \theta_{sat} \succeq \pmb H,
                           \qquad \bm{-1 & 0 & 0\\
                                      0  & 0 & 0 \\
                                      0  & 0 & 1} \pmb \theta_{sat} \geq 0 \notag\\
        \text{where, } & \qquad
        \pmb \Phi_{sat} = \bm{\pmb \phi_{sat}(1), \pmb \phi_{sat}(2), \hdots, \pmb \phi_{sat}(N-1)}^T \notag\\
        &\qquad \pmb H          = \bm{ \eta(2), \eta(3), \hdots, \eta(N)}^T \notag\\
        & \qquad \text{and $\pmb 1$ is a column vector of $1's$.} \notag
\end{align}
The experimental results indicate that the constraint on the parameter signs (\ref{eqn::theta_cond}) is not active for
the optimal values showing that the model structure is consistent with data.

The above estimate of $\theta_{sat}$ is the \itbf{maximum likelihood estimate (MLE) under exponential distribution for
model structure error} $\lr{\varepsilon_\eta}$  between the saturated system's response, $\eta_{sat}(k)$, and the actual
response of the catalyst, $\eta(k)$ (Appendix-\ref{sec::mle_derivation}). In the case of $NO_x$-sensor data with
cross-sensitivity, the model structure error includes the cross-sensitivity error $\varepsilon_{\chi}$ and is assumed to
be an exponential random variable.

Physically, the assumption of exponential distribution for $\varepsilon_\eta$ implies that the catalyst is saturated for
most of the time. Ideally this would be one of the objectives of the closed-loop urea-dosing controller as maximum
$NO_x$ reduction is achieved when the catalyst is saturated. The validity of the assumption on the error distribution is
verified experimentally in section-\ref{sec::epsilon_dist}.

\subsection{Distribution of the Parameter Estimate}

\subsection{Test-cell data results using FTIR Measurements (no cross-sensitivity)}
\subsection{Test-cell data results using $NO_x$ sensor measurement (cross-sensitivity)}
\subsection{Validating $\varepsilon_\eta$ distribution \label{sec::epsilon_dist}}
\subsection{Catalyst Mode Detection}
The saturated catalyst model's response is not dependent on the previous state but only on the inputs making the
response independent of the initial conditions. Thus, using the parameter estimates, a given data point can be
classified as saturated catalyst's response if the prediction error for the same inputs (inlet $NO_x$, $u_1$, and
mass flow rate, $F$) is less than the $\epsilon_{sat}$.
\begin{align}
        \abs{\hat \eta_{sat}(k) - \eta(k)} \leq \epsilon_{sat} \implies \text{Catalyst Saturation} \label{eqn::sat_cond}
\end{align}

A necessary condition for detecting the saturated segments of the data-set is that catalyst must have reached the
saturation with in the duration for a significant length of time in the data (Appendix-\ref{sec::necessary_cond}).
