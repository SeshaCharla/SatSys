\section{Maximum Likelihood Estimation of the Model Parameters}

The catalyst switches between saturation mode and normal mode based on states that are not completely controllable experimentally. The bounding condition on the saturated system's response (\ref{eqn::eta_bound}) can be used for detecting segments of the data where the catalyst's operation is normal or saturated. As the saturated catalyst's response forms a tight upper bound to the actual system's response, the parameters of the systems will be the solution to the linear programming problem that minimizes the area under the curve with the lower bound as the response of the actual system.
\begin{align}
\text{min. } \, \sum (\eta_{sat}(k)- \eta(k))^2\quad
\text{s.t } \, \lr{\eta_{sat}(k) \geq \eta(k)} \quad \forall  k \label{eqn::lin_opt_1}
\end{align}
%===
The above optimization problem is feasible as $\eta_{sat}, \eta \geq 0$ always. The solution provides the system's
parameter estimates under ideal conditions, i.e., in the absence of model structure uncertainties and measurement noise.
When the $NO_x$ sensor data is used, $\eta(k)$ is replaced with $\eta_y(k)$ as the bounding condition
(\ref{eqn::eta_y_bound}) still holds. Rewriting the optimization problem (\ref{eqn::lin_opt_1}) into a quadratic programming problem using regression vector (\ref{eqn::regression}) and introducing the constraints on the parameters (\ref{eqn::theta_cond}), we have the quadratic programming problem for estimating the parameters of the catalyst:
\begin{align}
        \text{minimize} &\qquad \norm{{\Phi_{sat} \theta_{sat} - H}}^2      \label{eqn::lin_prog}\\
        \text{subject to} &\qquad \Phi_{sat} \theta_{sat} \succeq H,
                           \qquad \bm{-1 & 0 & 0\\
                                      0  & 0 & 0 \\
                                      0  & 0 & 1} \pmb \theta_{sat} \geq 0 \notag\\
        \text{where, } & \qquad
        \Phi_{sat} = \bm{\phi_{sat}(1), \phi_{sat}(2), \hdots, \phi_{sat}(N-1)}^T \notag\\
        &\qquad H = \bm{ \eta(2), \eta(3), \hdots, \eta(N)}^T \notag\\
        & \text{$\pmb 1$ is a column vector of $1's$.} \notag
\end{align}
The experimental results indicate that the constraint on the parameter signs (\ref{eqn::theta_cond}) is not active for
the optimal values showing that the model structure is consistent with data.

The above estimate of $\theta_{sat}$ is the \itbf{maximum likelihood estimate (MLE) under half-normal distribution for
model structure error} $\lr{\varepsilon_\eta}$  between the saturated system's response, $\eta_{sat}(k)$, and the actual
response of the catalyst, $\eta(k)$ \cite{942636} (Appendix-\ref{sec::mle_derivation}). In the case of $NO_x$-sensor
data with cross-sensitivity, the model structure error includes the cross-sensitivity error $\varepsilon_{\chi}$ and is
assumed to be an exponential random variable. The assumption of half-normal distribution for $\varepsilon_\eta$ implies
that the catalyst close to saturated for most of the time. Ideally this would be one of the objectives of the
closed-loop urea-dosing controller as maximum $NO_x$ reduction is achieved when the catalyst is saturated.


% The validity of the assumption on the error distribution is verified experimentally using goodness of fit for half-normal distribution demonstrated by the p-values \cite{kojadinovic2012goodness} in section-\ref{sec::epsilon_dist}.

\subsection{Distribution of the Parameter Estimate}
Using asymptotic properties of MLE (Theorem-7.3 in \cite{kay1993estimation}), with significantly large data lenght $(N)$ the parameter estimate $\hat \theta_{sat}$ asymptotically follows a normal distribution.
\begin{align}
        \hat \theta_{sat} \sim \mathcal{N} \lr{\theta_{sat}, I^{-1}(\theta_{sat})}
\end{align}

% \subsection{Variance in $\eta_{sat}$ prediction}
The variance in the $\eta_{sat}$ prediction for the given time-step is obtained by propagating the uncertainity of paramter estimates (\ref{eqn::theta_dist}). Using, (\ref{eqn::regression}), we have,
\begin{align}
        \Var\lr{\eta_{sat}(k)} = \sigma^2_{\eta_{sat}}(k) &= \phi_{sat}(k)^T I^{-1}(\theta_{sat}) \phi_{sat}(k)
\end{align}

\subsection{Test-cell data results using FTIR Measurements (no cross-sensitivity)}
The parameter estiamtes for different catalysts of same capacity under similar experimental conditions (RMC, hot and cold FTP) are identical demonstrating the validity of the modelling and paramter estimation approach. The predicted $NO_x$ reduction under catalyst saturation is equvalent considering the uncertainity.

\subsection{Test-cell data results using $NO_x$ sensor measurement (cross-sensitivity)}
Further the uncertainity bounds are widened when the $NO_x$ sensor measurements are used for estimating the paramters of the saturated system.

\subsection{Validating $\varepsilon_\eta$ distribution \label{sec::epsilon_dist}}
The goodness of fit of half-normal distribution for $\lr{\varepsilon_{\eta}}$ is demonstrated by the p-values \cite{kojadinovic2012goodness}.

% \subsection{Limitations of the Parameter Estiamtion Method}
A limitation of the delineated parameter estimation method is that the data-set must have response of the catalyst under
saturation for finite duration. If this condition is not met, the algorithm estimates the paramters considering the
maximum concentration of adsorbed ammonia as the concentration corresponding to catalyst saturation (Appendix-\ref{sec::necessary_cond}).

