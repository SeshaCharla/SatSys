\section{Catalyst Mode Detection}
The saturated catalyst model's response is not dependent on the previous state but only on the inputs making the
response independent of the initial conditions. Thus, using the parameter estimates, a given data point can be
classified as saturated catalyst's response if the prediction error for the same inputs (inlet $NO_x$, $u_1$, and
mass flow rate, $F$) is less than the $\epsilon_{sat}$.
\begin{align}
        \abs{\hat \eta_{sat}(k) - \eta(k)} \leq \epsilon_{sat} \implies \text{Catalyst Saturation} \label{eqn::sat_cond}
\end{align}
For the current test-cell  and truck data set, $\epsilon_{sat} = 2.5\times 10^{-3} \, mol/m^{-3}$.
\begin{figure}[H]
        \centering
        \includegraphics[width=\figWidth]{./figs/4_cat_mode/SatDetect_dg_rmc_1_FTIR.png}
        \caption{Catalyst Mode Detection for DG RMC 1 Data}
        \label{fig::cat_mode_det}
\end{figure}
The catalyst mode detection for DG RMC 1 test-cell data using FTIR measurements is shown in
Figure-\ref{fig::cat_mode_det}. The catalyst is found to be close to saturation (or at its maximum $NO_x$ reduction)
when the mass flow rate and the urea dosing rates are high, consistent with the physical understanding of the catalyst
operation. Further, the flow rate can introduce virtual cieling on the maximum achievable $NO_x$ reduction due to
reduced residence time in the catalyst which would be indistinguishable from saturated catalyst behaviour as the $NO_x$
reduction would be independent of urea dosing.
