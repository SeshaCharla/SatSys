\section{Aging Detection from Truck Data}
%===
The performance of the proposed aging detector is also demonstrated using the $NO_x$ sensor data from four long-haul
trucks. The road data was collected from trucks operating in the United States for two days separated by a few years
(Table~\ref{tab::truck_data_summary}) during which the catalyst has aged based on the existing performance measures. The
data contains measurements of $NO_x$ sensors before and after the catalyst, as well as other required variables
including flow rate, temperature and urea injection rate. The data preprocessing includes,
\begin{inparaenum}
        \item removing the rows with missing values,
        \item removing the rows whose operating temperature range is beyond range of $200-300 \,^0C$,
        \item interpolating the missing data if the data breaks are smaller than a minute, and finally,
        \item smoothing the data using a non-causal Chebyshev filter.
\end{inparaenum}
The preprocessed data is then partitioned into drive segments with no breaks which correspond to a continuous driving
period from engine start to engine stop.
%===
% Truck Maping
% Truck A - AD Transport (adt)
% Truck B - Mesilla Valley (mes)
% Truck C - Werner (wer)
% Truck D - Transwest (trw)
% =====
\begin{table}[ht]
        \centering
        \caption{Truck Data Set Years and Mileage Difference}
        \label{tab::truck_data_summary}
        \begin{tabular}{c c c c}
        \hline \hline
         Truck & Earlier Data & Latter Data & Mileage \\
               &  Year            & Year    & Difference\\ \hline \hline
        Truck A & $2015$ & $2017$ & $2.8 \times 10^5$\\
        Truck B & $2015$ & $2018$ & $11.9 \times 10^5$\\
        Truck C & $2015$ & $2017$ & $5.9 \times 10^5$\\
        Truck D & $2015$ & $2016$ & $6.6 \times 10^5$ \\
        \hline \hline
        \end{tabular}
\end{table}
%===
Unlike the test-cell data where the engine drive-cycles are repeatable and produce a significant fraction of samples where the catalyst is close to saturation, the truck data drive segments constitute varied driving conditions with fewer samples $\lrf{\leq 10\%}$ that are close to catalyst saturation. A necessary condition for the proposed parameter estimation and aging detection framework is that the data contains a sufficiently large number of samples close to catalyst saturation such that the asymptotic properties of the MLE and the test statistic hold (\ref{eqn::Ns_min}). Hence, the drive segment data is further pruned to only include segments with at least $150$ samples ($120$ in the case of Truck C) close to catalyst saturation.
\begin{align}
        N_{sat} &\geq N_{sat}^{min} \label{eqn::Ns_min} \\
        \beta_{truck} &= 250 \label{eqn::beta_truck}
\end{align}
The threshold of the Wald test statistic (\ref{eqn::beta_truck}) for aging detection in trucks is chosen such that it maximizes the probability of detection while minimizing the probability of false alarm and missed detection for the given limited truck data. A larger data set would help arrive at the thresholds using Neyman-Pearson criteria \cite{kay1998detection}. The value of the test statistic for the drive segments for each truck is tabulated in Tables~\ref{tab::adt_aging_res}, \ref{tab::trw_aging_res}, \ref{tab::mes_aging_res}, and \ref{tab::wer_aging_res}. A summary of the aging detection results is tabulated in Table~\ref{tab::truck_aging_summary}.
%===
\begin{table}[ht]
        \centering
        \caption{Summary of Aging Detection Results on Truck Data}
        \label{tab::truck_aging_summary}
        \begin{tabular}{c c c c c}
        \hline \hline
         Truck & Total Drive & Detection & False  & Missed \\
               & Segments    &         & Alarm & Detection  \\\hline \hline
        Truck A   & 8  & 4 & 0 & 4 \\
        Truck B   & 11 & 9 & 1 & 1 \\
        Truck C   & 9  & 7 & 1 & 1 \\
        Truck D   & 8  & 6 & 2 & 0 \\
        \hline \hline
        \end{tabular}
\end{table}
The detector performs reasonably well on three of the four trucks with high probability of detection and low false alarm
and missed detection rates. Truck A has relatively lower mileage difference between the earlier and latter data years
which may explain the lower detection performance. The performance is limited by the amount of data available with
sufficient samples close to catalyst saturation. Further, the lack of ground truth on the actual individual catalyst's
aging levels limits the ability to quantify the performance of the detector. Nevertheless, the results demonstrate the
potential of the proposed aging detection framework for on-board application using $NO_x$ sensor data.
