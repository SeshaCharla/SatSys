\section{Conclusion}
%===
The dynamics of $NO_x$ reduction in diesel engine SCR systems under catalyst saturation were used to develop an aging
detector based on the hypothesis that the model parameters associated with catalyst storage capacity change as the
catalyst ages. A dynamic model for $NO_x$ reduction under saturation was formulated from molar conservation while
explicitly accounting for the interplay between the residence time of the reacting flow and the sampling time of the
sensors. A maximum-likelihood parameter estimation algorithm was proposed using the bounding condition between the
actual system response and the response under catalyst saturation. The algorithm was validated using experimental test
cell data under RMC and Hot-FTP conditions. The $NO_x$-sensor's ammonia cross-sensitivity was shown to introduce a bias
in the parameter estimates, reducing the separation between the estimates for degreened and aged catalysts. A
hypothesis-testing framework based on the Wald test was then developed for catalyst aging detection using the estimated
parameters. The detector's performance was demonstrated using test cell data at different aging levels and real-world
data from four long-haul trucks. The results indicate that the proposed framework can reliably identify catalyst aging
even in the presence of sensor cross-sensitivity and varying operating conditions, provided that the data include a
sufficient number of samples near the saturation regime, which forms a necessary condition for aging detection.
